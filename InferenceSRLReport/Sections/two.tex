\section{Build an Inference Engine: Pipeline}

The pipeline has been implemented in Python3. The README is in the same directory as the code and will explain how the code should be run. The WMC computation has been carried out in miniC2D whereas gvpr (graph pattern scanning and processing language) has been used to obtain the vtree statistics using de dot file created by miniC2D. The results are shown in tables~\ref{table:results-task11} to~\ref{table:results-bayesian}. We use the default flag settings for the vtrees because they perform quite well generally. Sometimes there are other vtree flags that perform better, as mentioned in Section~\ref{subsec:knowledgecompilation}, but the difference is negligible. Any of the vtrees that are compact, which corresponds to a low width, is a good candidate.
\\[2ex]
We used ENC2, as specified in the first task, as our encoding. We chose this because it minimizes the amount of variables needed to get a full CNF. The encoding, as we implemented it, is not always very efficient. In the case of task 1.2 we have more variables than actually needed. Table~\ref{table:results-task12} shows that we need 16 variables and 28 clauses to represent the CNF for \texttt{query(path15)}. In the CNF folder where, where we manually created the CNFs, we only need 6 variables and 8 clauses. The difference can be explained as we created two lambda variables and a rho variable for every atom in ENC2. However, edges are never used in their negation in clauses. This means that there are no rules that express \texttt{$\backslash$+edge(1,3)} for example. The lambda that expresses \textit{not} an edge and the corresponding $\rho$ are actually useless. A possible way to fix this is to scan over all the rules and ensure that the negation never occurs (or that only the negation occurs).
\\[2ex]
\begin{table}[h]
\centering
\caption{Results task11}
\label{table:results-task11}
\resizebox{\textwidth}{!}{\begin{tabular}{l|l|l|l|l|l|l|l|l|l|l|l}
\textbf{query} & \textbf{probability}         & \textbf{count time (s)} & \textbf{vtree time (s)} & \textbf{CNF time (s)} & \textbf{\#vars CNF} & \textbf{\#clauses CNF} & \textbf{depth vtree} & \textbf{branching factor vtree} & \textbf{\#edges circuit} & \textbf{\#nodes circuit} & \textbf{total runtime (s)} \\ \hline
burglary       & $\frac{0.364}{0.372} = 0.98$ & 0.006                   & 0.001                   & 0.000                 & 24                  & 37                     & 13                   & 2                               & 28                       & 29                       & 0.021                      \\ \hline
e(none)        & $\frac{0.290}{0.371} = 0.78$ & 0.005                   & 0.000                   & 0.000                 & 24                  & 37                     & 13                   & 2                               & 28                       & 29                       & 0.020                     
\end{tabular}}
\end{table}


For tasks with no evidence, we only need to compute the model with the query. The number of counted models equals the probability of the query. If there is evidence present, then we need to perform two computations: one with the query and one without the query. Dividing the joint probability (query and evidence $P(Q,E)$) by the joint probability of the conditionals $P(E)$ equals the conditional probability we are looking for.
\\\\
\begin{table}[h]
\centering
\caption{Results task12}
\label{table:results-task12}
\resizebox{\textwidth}{!}{\begin{tabular}{l|l|l|l|l|l|l|l|l|l|l|l}
\textbf{query} & \textbf{probability} & \textbf{count time (s)} & \textbf{vtree time (s)} & \textbf{CNF time (s)} & \textbf{\#vars CNF} & \textbf{\#clauses CNF} & \textbf{depth vtree} & \textbf{branching factor vtree} & \textbf{\#edges circuit} & \textbf{\#nodes circuit} & \textbf{total runtime (s)} \\ \hline
path15         & 0.258                & 0.000                   & 0.000                   & 0.000                 & 16                  & 28                     & 10                   & 2                               & 19                       & 20                       & 0.015                      \\  \hline
path16         & 0.217                & 0.002                   & 0.001                   & 0.000                 & 22                  & 55                     & 13                   & 2                               & 27                       & 28                       & 0.017                     
\end{tabular}}
\end{table}

The results of the big Bayesian network are shown in table~\ref{table:results-bayesian}. We have definitely succeeded in topping problog in terms of computational speed. One of the leaf nodes at the bottom of the graph is \texttt{hREKG}. This node needed about 10 to 15 minutes to compute in problog, but it was computed by miniC2D in 100 milliseconds on average. Our encoding does not suffer from the overload in variables as in task 1.2 because each atom occurs both in its normal form and its negation in the rule bodies. However, the computed probabilities seem to fluctuate around the result that problog achieves. The computed answer of miniC2D differs every time we run it, but it approximates the correct answer quite closely. We do not know the exact reason for this, but we have some speculations:
\begin{itemize}
\item MiniC2D uses rounding frequently for computed intermediate results. The errors cascade and become larger when the amount of variables increase enough.
\item We use dictionaries and sets in our program to map logic and remove duplicate elements. This may cause the ordering in which clauses and variables get selected to become random. In conjunction with the rounding problem above, this means that the errors do vary indeed.
\item We made an implementation mistake somewhere that only occurs for the Bayesian network somehow.
\end{itemize}

\begin{table}[h]
\centering
\caption{Results task Bayesian network}
\label{table:results-bayesian}
\resizebox{\textwidth}{!}{\begin{tabular}{l|l|l|l|l|l|l|l|l|l|l|l}
query       & \textbf{probability} & \textbf{count time (s)} & \textbf{vtree time (s)} & \textbf{CNF time (s)} & \textbf{\#vars CNF} & \textbf{\#clauses CNF} & \textbf{depth vtree} & \textbf{branching factor vtree} & \textbf{\#edges circuit} & \textbf{\#nodes circuit} & \textbf{total runtime (s)} \\ \hline
hREKG(LOW)  & 0.178                & 0.039                   & 0.015                   & 0.001                 & 319                 & 482                    & 32                   & 2                               & 420                      & 421                      & 0.070                      \\ \hline
pRESS(HIGH) & 0.484                & 0.008                   & 0.003                   & 0.000                 & 83                  & 130                    & 22                   & 2                               & 104                      & 105                      & 0.026                      \\ \hline
aRTCO2(LOW) & 0.193                & 0.023                   & 0.006                   & 0.000                 & 179                 & 248                    & 28                   & 2                               & 228                      & 229                      & 0.045                      \\ \hline
sAO2(LOW)   & 0.770                & 0.027                   & 0.007                   & 0.000                 & 214                 & 302                    & 30                   & 2                               & 274                      & 275                      & 0.050                     
\end{tabular}}
\end{table}
\vspace{5mm}
We have been going on a wrong trail for a very long time, which caused us to lose precious time.  For $\sim$60 hours we have been dabbling in the problog source code to figure out a way to generate a CNF that fits our encoding. We got on the right track after the discussion the week before the deadline. Due to this, we had to implement some constraints as a shortage of time did not allow us to streamline everything:
\begin{itemize}
\item All clauses must be supplied (even the 0 probability ones). In task11, there is a missing clause for \textit{not burglary} ad \textit{no earthquake}. It is possible to sort variables into categories and for each rule to loop over all the combinations of the body. From there we can verify which ones are missing and assign a probability of 0 to them. This required quite some time which we did not have, so we kept it as a constraint.
\item We assume in the case of multiple headers, that they are on the same line (and not spread over multiple lines). Looking for all rules that have the exact same body and merging the headers together (separated by a semicolon) will fix this.
\item We assume the evidence is always true. If this is not the case, the evidence list must not just accept the names of the evidence, but a tuple of \verb|(<name>,<boolean)|. The getOrderedWeights method must be adjusted accordingly.
\item \textbf{Task23 does not work.}  Queries from task23 must be grounded before we can loop over them, but we can only verify one query at a time.  First ground the program, extract all the queries and store them in a list. Then we replace the query line with one of the stored queries at a time. This will means that the queries need to be manually stored every time the program changes, but it would be a possible solution.
\item We keep the number of variables to a minimum by replacing occurrences of rules with headers with probability 1 into the body of other rules. This approach works fine for task12, but not for task23 where we may still need those rules. The solution for this would be to see if the rules are still needed (for query or evidence purposes). 
\end{itemize}

The constraints and the solutions for it are explained in more detail in \texttt{cnf\_converter.py} in the source code.


