\section{Comparison with Approximate Inference}

\subsection{SampleCount}

Variables with binary values can be set to true or false through logical assertion. If $A$ and $B$ cannot be both true and we know that $A$ is true, then $B$ must be false. The most complete form of propagation is done by generating all possible models and to assert that: (i) a variable $I$ is certainly true if $I$ is true in every model, and (ii) a variable $J$ is certainly false if $J$ is false in every model. Every other variable is in an \textit{unknown} state, which can be either true or false.
\\\\
The \texttt{approxcount} function in \textbf{SampleCount} allows us to set a threshold for which a variable is unknown. For example, the \texttt{-dont\_care\_fract N} option considers variables that are positive in (50-N)\% to (50+N)\% samples to be dont_cares. If N equals 25, then any variable that has a true value between 25\% and 75\% is unknown. A variable that is positive more than 75\% of the samples is certainly true, and certainly false if it is positive less than 25\% of the samples.
\\\\
Having this threshold in place means that variables are set to certainly true, even if there are (relatively) few samples that are actually false. This allows a higher degree of propagation and will improve computational speed. The drawback is the loss of accuracy of the results. A model that may not have been considered a valid solution because of a variable that was supposed to be false, may be suddenly true because it is true in the majority of models. This model will then be added to the count of the query while it shouldn't have. This reasoning also holds in the other direction: a model that should have been counted as a solution may be considered invalid.
\\\\
Inserting this functionality in the pipeline is not very difficult. The execution path of miniC2D in \texttt{run.py} should be replaced by the path to \texttt{./approxcount} and should accept the generated cnf file as an argument. The appropiate flags can be passed along in the run method in case this inference engine is selected. Note that if no inference engine is entered, then the results will be computed by problog. The result of the approxcount are sampled solutions, but these do not contain any weights. To compute the weighted models, the weights must still be added to each literal. MiniC2D then computes the conditional probability for a given query.


\subsection{Search Tree Sampler}

The \textbf{SearchTreeSampler (STS)} is a sampling technique. The search tree is explored uniformly in a breadth-first way. At each level, a subset of the representative nodes are subsampled. The samples can be used to estimate the model count. So instead of counting all models, it takes a subset of each subtree to generalize over. As with the SampleCount, the accuracy of the result is sacrificed for computational speed, albeit using a different method.
\\\\
Running this is rather straightforward: only the cnf file is required to return all the samples and corresponding values for the variables. Every level corresponds with a variable, so if there are 319 variables then there are 319 levels. This is not supported in our pipeline because the output is in a completely different format and would require too much time for our implementation to add. 
